\chapter{Conclusions}

Regional climate is not well captured by global climate models (GCMs) and global reanalysis datasets which are employed at coarse resolutions. However, dynamic processes at unrepresented scales are significantly drivers for regional and local climate variability especially over complex terrain \cite{soares2012wrf}. In order to capture those fine-scale dynamical features, high horizontal resolution is needed to allow a more accurate representation of fine scale forcing, and the better representation of processes and interactions, as former studies have already showed \cite{leung2003regional, rauscher2010resolution}. Also, better represented regional climate information can lead to effective action for responses to climate change and mitigation of negative impacts taken by local stakeholders and policymakers.

In order to model regional climate at a higher spatial and temporal resolution over a limited area, downscaling methods have been developed including nested limited-area models (LAMs) and variable-resolution (including stretched-grid) global climate models (VRGCMs) \cite{laprise2008challenging}. LAMs are more commonly referred as regional climate models (RCMs) when applying to climate scales. RCMs are forced by output of GCMs or reanalysis data, and have been widely used \cite{christensen2007regional, bukovsky2009precipitation, caldwell2009evaluation, mearns2012north}. Nudging is used in RCMs to overcoming the inability of representing large scale features \cite{laprise2008regional}.

Over the past decade, variable-resolution global climate models (VRGCMs) have been introduced as an alternative way for studying regional climate and applications \cite{fox1997finite, fox2006variable, ringler2008multiresolution, skamarock2012multiscale, rauscher2013exploring, zarzycki2015effects}. Compared with RCMs, a key advantage of VRGCMs is that they use a single, unified modeling framework, rather than two separate models (GCM and RCM) with potentially disparate dynamics and physics parameterizations. VRGCMs also provide a cost-effective method of reaching high resolutions over a region of interest -- the limited area simulations in this study at 0.25$^\circ$ and 0.125$^\circ$ resolution represent a reduction in required computation of approximately 10 and 25 times, respectively, compared to analogous globally uniform high-resolution simulations. 

%Part 1
This study has investigated the variable-resolution Community Earth System Model (VR-CESM) for two-way dynamically downscaled climate modeling. VR-CESM was evaluated for modeling California's unique regional climate and compared against gridded observational datasets, reanalysis data and the WRF model (forced with ERA-Interim data at lateral boundaries). Based on 26 years of high-resolution historical climate simulations (1980-2005), we analyzed the mean climatology of California across its climate divisions in terms of both near-surface temperature and precipitation. Generally, when compared with gridded observational datasets, both VR-CESM and WRF adequately represented regional climatological patterns with high spatial correlations ($>$0.94). Uncertainty between reference datasets is apparent, and is statistically significant over some climate divisions, making it necessary to utilize more than one high-quality observational product in the model evaluation. Overall, we found that VR-CESM showed comparable performance to WRF for regional climate modeling at spatial resolutions of 10-30 km.

In summary, VR-CESM demonstrated competitive utility for studying high-resolution regional climatology when compared to a regional climate model (WRF). Compared to regional models, variable-resolution models are more suitable for regional climate studies where non-local processes are a major influence, including two-way interactions at the nest boundary and potential upstream impacts \cite{sakaguchi2015exploring}.  Variable-resolution models are also useful for assessing and tuning resolution dependence of physical parameterizations in global models, and are also valuable for short-term weather prediction \cite{zarzycki2015experimental}. On the other hand, RCMs tend to have more sub-grid parameterization choices that can be tailored for particular studies (e.g., \cite{cassano2011performance}) and tend to be more efficient, as computational expense can be precisely targeted. Deviations exhibited within these models are not indicative of deep underlying problems with the model formulation, but one should nonetheless be aware of these biases when using these models for climate studies. This study suggests that VRGCMs are, in general, useful tools for assessing climate change over the coming century. As the need for assessments of regional climate change increases, alternative modeling strategies, including VRGCMs will be needed to improve our understanding of the effects of fine-scale processes representation in regional climate regulation. Future work will focus on the capability of the variable resolution system to correctly capture the features of discrete, extreme heat and precipitation events.

%Part 2
VR-CESM is applied to understand the impact of irrigation on the regional climate of California. Irrigation is an important contributor to the regional climate of heavily irrigated regions, and within the U.S. there are few regions that are as heavily irrigated as California's Central Valley, responsible for 25$\%$ of domestic agricultural products \cite{wilkinson2002potential}. However, irrigation effects are usually ignored in climate models for several reasons: irrigation usually occurs over a relatively small area ($\sim$2$\%$ of global land surface) and produces a seemingly negligible cooling effect compared to global greenhouse warming \cite{boucher2004direct}. Nonetheless, irrigation is a potentially important factor in regulating climate patterns at regions scales, where there is a growing need for accurate climate assessments and projections.

A flexible irrigation scheme with relatively realistic estimates of agricultural water use is employed and the impact of irrigation on mean historical climatology and heat extremes is investigated. We have found that the cooling effect caused by irrigation was obvious in the Tmax field, which arose from the greatly increased latent heat flux associated with daytime ground evaporation. With irrigation enabled, an exceptional warm bias associated with a long forward tail of the frequency distribution of Tmax is alleviated, although a slight cold bias remained at higher elevations. Further, the cooling effect associated with irrigation led to a reduction in length and frequency of hot spells for about 20$\%$ and 30$\%$, closely matched to observations, and a decrease in the heat stress frequency by about 22$\%$ for cropland. To summarize, irrigation in the CV is an important component of the region's surface energy budget that must be parameterized in high-resolution climate models in order to properly simulate temperature statistics.

%Part 3
There is substantial and growing interest in understanding the character of precipitation within a changing climate, motivated largely by its pronounced impacts on water availability and flood management in both human and natural systems \cite{hegerl2004detectability, kharin2007changes, scoccimarro2013heavy}. Among past studies addressing precipitation, extremes have been a major focus, particularly drought and flood events \cite{seneviratne2012changes}. Future climate projections, particularly those addressing the frequency and intensity of rare events, are inevitably subject to large uncertainties. We have investigated the changing character of precipitation frequency and intensity in the western United States (WUS) over the 21st century, as predicted from long-term ensemble runs conducted with VR-CESM with a fine grid resolution. The WUS is known to be particularly vulnerable to hydrological extremes, particularly floods and droughts \cite{leung2003hydroclimate, caldwell2010california}, and hosts a variety of local features and microclimates associated with its rough and varied topography.  Simulations of the future climate are performed in accordance with the representative concentration pathway (RCP) 8.5 scenario, which describes a ``business-as-usual'' projection for GHGs \cite{riahi2011rcp}.

Evaluated against historical gridded observations and reanalysis data, VR-CESM was found to accurately capture the spatial patterns of precipitation, including precipitation frequency and intensity, although it exhibited an overestimation of precipitation over the eastern flank of the Cascades, throughout California's Central Valley and along the Sierra Nevada.  Nonetheless, there was clear improvement in the representation of precipitation features when compared with coarse 1$^\circ$ resolution simulations. Both mean changes to precipitation and distributions of both non-extreme and extreme events, projected by the VR-CESM model under climate forcing, have been investigated.  Although constrained by water influx and soil moisture, changes to extreme precipitation are hypothesized to follow the C-C relationship more closely than total precipitation amount ($\sim$7\% per degree K). It is found that continental evaporation and oceanic water vapor transport are insufficient vapor sources to maintain RH levels above a certain threshold temperature. In response, mean precipitation increase is fairly inhomogeneous, with a more pronounced increase in the Northwest where vapor transport is enhanced. The impacts of ENSO are wide-reaching, with a statistically significant response observed in the character of precipitation throughout California, the intermountain west and the southwest regions, as well as impacting mean precipitation through the Cascades.

%In general, this only seemed to be the case over the intermountain west; the northwest exhibited an enhanced response from extreme precipitation ($\sim$10\% per degree K), whereas California and the southwest observed essentially no response.

%IMG_1387.jpg IMG_1388.jpg to be added

