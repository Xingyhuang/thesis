\chapter{Introduction}

\section{Background}

Global climate models (GCMs) have been widely used to simulate both past and future climate. Although these models have demonstrable success in representing large-scale features of the climate system, they are usually employed at relatively coarse resolutions ($\sim$1$^\circ$), largely as a result of the substantial computational cost required at higher resolutions. Global climate reanalysis datasets, which assimilate climate observations using a global model, represent a best estimate of historical weather patterns. However, reanalysis datasets still cannot fulfill the needs of policymakers, stakeholders and researchers that require high-resolution regional climate data (\url{http://reanalyses.org/atmosphere/overview-current-reanalyses}). Regional features such as microclimates, land cover, and topography, are not well captured by either GCMs or reanalysis datasets \cite{leung2003regional}.However, dynamical processes at unrepresented scales are significant drivers for local climate variability, especially over complex terrain \cite{soares2012wrf}. In order to capture fine-scale dynamical features, high horizontal resolution is needed for a more accurate representation of small-scale processes and interactions \cite{rauscher2010resolution}. With these enhancements, regional climate data is expected to be more useful for formulating climate adaptation and mitigation strategies locally.

In order to model regional climate at high spatial resolutions over a limited area, downscaling techniques have been developed, such as statistical and dynamical downscaling. Dynamical downscaling typically uses nested limited-area models (LAMs) or, more recently, variable-resolution enabled GCMs (VRGCMs) \cite{laprise2008regional}. In this context, LAMs are typically referred to as regional climate models (RCMs) when used for climate study. Forced by the output from GCMs or reanalysis datasets, RCMs have been widely used to capture physically consistent regional and local circulations at the needed spatial and temporal scales \cite{christensen2007regional, bukovsky2009precipitation, mearns2012north}. Recently, there has been a growing interest in the use of VRGCMs for modeling regional climate. Unlike RCMs, VRGCMs use a relatively coarse global model with enhanced resolution over a specific region \cite{staniforth1978variable, fox1997finite}.  Strategies that have been employed for transitioning between coarse and fine-resolution regions within a VRGCM include grid stretching \cite{fox1997finite, mcgregor2008updated} and grid refinement \cite{ringler2008multiresolution, skamarock2012multiscale, zarzycki2014aquaplanet}. VRGCMs have demonstrated utility for regional climate studies and applications at a reduced computational cost compared to uniform-resolution GCMs \cite{fox2006variable, rauscher2013exploring, zarzycki2015effects}. 

Compared with RCMs, a key advantage of VRGCMs is that they use a single, unified modeling framework, rather than two separate models (GCM and RCM) with potentially disparate dynamics and physics parameterizations. RCMs may suffer from potential inconsistencies between the global and regional scales and lack two-way interactions at the nest boundary \cite{warner1997tutorial, mcdonald2003transparent, laprise2008challenging, mesinger2013limited}, which can be mitigated with the use of VRGCMs. VRGCMs also provide a cost-effective method of reaching high resolutions over a region of interest -- the limited area simulations in this study at 0.25$^\circ$ and 0.125$^\circ$ resolution represent a reduction in required computation of approximately 10 and 25 times, respectively, compared to analogous globally uniform high-resolution simulations. For the purposes of this paper, we focus on the recently developed Community Earth System Model with variable-resolution option (VR-CESM) as our VRGCM of interest. This configuration is driven by the Community Atmosphere Model's (CAM's) Spectral Element (SE) dynamical core, which possesses attractive conservation and parallel scaling properties \cite{dennis2011cam, taylor2011conservation}, as well as recently developed variable-resolution capabilities \cite{zarzycki2014aquaplanet, zarzycki2015experimental}. This model has been employed by \cite{zarzycki2014using} to show that a high-resolution refinement patch in the Atlantic basin for simulating topical cyclones represented significant improvements over the unrefined simulation. \cite{zarzycki2015effects} also compared the large-scale climatology of VR-CESM 0.25$^\circ$ and uniform CESM at 1$^\circ$, and found that adding a refined region over the globe did not noticeably affect the global circulation. \cite{rhoades2015characterizing} has also assessed the use of VR-CESM for modeling Sierra Nevada mountain snowpack in the western United States.

Irrigation effects are usually ignored in climate models for several reasons: irrigation usually occurs over a relatively small area ($\sim$2$\%$ of global land surface) and produces a seemingly negligible cooling effect compared to global greenhouse warming \cite{boucher2004direct}. Nonetheless, irrigation is a potentially important factor in regulating climate patterns at regions scales, where there is a growing need for accurate climate assessments and projections. Past studies have typically addressed the climatic effects of irrigation in limited-area models (LAMs) \cite{snyder2006regional, kueppers2007irrigation}, which in the context of climate modeling are typically referred to as regional climate models (RCMs). In these studies, irrigation is modeled by accounting for the amount of irrigated water needed and the area of cropland where irrigation is applied. Using a multi-model ensemble of RCM simulations, \cite{kueppers2008seasonal} found that the behavior of RCMs varied in representing effects of irrigation on regional climate, depending on each model's physics, as well as on the configuration of the irrigation parameterization.

California is the most irrigated state in the U.S., and most of California's irrigated cropland is distributed over the Central Valley (CV), which is responsible for 25$\%$ of domestic agricultural products \cite{wilkinson2002potential}. Irrigation is an important contributor to the regional climate of heavily irrigated regions, and within the U.S. there are few regions that are as heavily irrigated as California's Central Valley, responsible for 25$\%$ of domestic agricultural products. In order to model regional climate over the CV, relatively fine horizontal resolution is needed to more accurately represent microclimates, land-use, small-scale dynamical features and corresponding interactions \cite{leung2003regional, rauscher2010resolution}.  There is a need to study the impact of irrigation on regional climate over the CV, based on VR-CESM, which features a more flexible irrigation scheme with relatively realistic estimates of regional agricultural water use.

There is substantial and growing interest in understanding the character of precipitation within a changing climate, motivated largely by its pronounced impacts on water availability and flood management in both human and natural systems \cite{hegerl2004detectability, kharin2007changes, scoccimarro2013heavy}.  Among past studies addressing precipitation, extremes have been a major focus, particularly drought and flood events \cite{seneviratne2012changes}.  Overall, it is widely agreed that although atmospheric water vapor concentration is increasing, the impacts of a changing climate on the character of precipitation is far more complicated.  Extreme precipitation events are particularly nuanced:  Our best projections suggest that extreme precipitation events will intensify even in regions where mean precipitation decreases \cite{tebaldi2006going, kharin2007changes}.

Although several past studies have investigated climate extremes at the global scale \cite{seneviratne2012changes}, studies addressing extremes at local and regional scales are less common. It is well understood how increased GHG concentrations have contributed to the observed intensification of heavy precipitation events over the tropical ocean \cite{allan2008atmospheric} and the majority of Northern Hemisphere overland areas \cite{min2011human}, but changes are much more poorly understood at regional scales where meteorological variability is large \cite{trenberth2011changes}. This issue of insufficient regional-scale climate information has been a major outstanding problem in climate science, as stakeholders and water managers typically require fine-scale information on climate impacts in order to effectively develop adaptation and mitigation strategies. 

The western United States (WUS) area is known to be particularly vulnerable to hydrological extremes, particularly floods and droughts \cite{leung2003hydroclimate, caldwell2010california}, and hosts a variety of local features and microclimates associated with its rough and varied topography. It is important to understand the changes in the character of precipitation, in terms of frequency and intensity, from recent history through the end of the 21st century over WUS.

\section{Outline of Thesis}

In this thesis, regional climate has been studied from past to future over western United States (especially, California), working with VR-CESM--the newly developed variable-resolution enabled Community Earth System Model. This thesis is organized as follows. In Chapter 2, VR-CESM is assessed for long-term regional climate modeling over California against a traditional RCM -- the Weather Research and Forcasting (WRF) model. We aim to fill that gap by analyzing the performance of VR-CESM against gridded observational data, reanalysis product and in comparison to a traditional RCM forced by reanalysis data. This chapter focuses on the models' ability to represent current climate statistics, particularly those relevant to heat and precipitation extremes. In Chapter 3, VR-CESM is further applied to understand the impact of irrigation on the regional climate of California. A flexible irrigation scheme with relatively realistic estimates of agricultural water use is employed and the impact of irrigation on mean historical climatology and heat extremes is investigated. In Chapter 4, the projected changing character of precipitation in the western United States over the 21st century has been investigated under the RCP 8.5 ``business-as-usual'' scenario. Both mean changes to precipitation and distributions of both non-extreme and extreme events, projected by the VR-CESM model under climate forcing, have been studied. A concise conclusion is given in the last chapter.
